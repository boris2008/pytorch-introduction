%\PassOptionsToPackae{gray}{xcolor}
\documentclass[hyperref={pdfpagelabels=false},12pt]{beamer}
\setbeamertemplate{frametitle}[default][center]
\mode<presentation>
{
 \usetheme{Warsaw}      % or try Darmstadt, Madrid, Warsaw, ...
 \usecolortheme{default} % or try albatross, beaver, crane, ...
 \usefonttheme{default}  % or try serif, structurebold, ...
 \setbeamertemplate{footline}[frame number]
 \setbeamertemplate{caption}[numbered]
}

\usepackage[utf8]{inputenc}
\usepackage{mathtools}
\usepackage{bm}
\usepackage{helvet}
\usepackage{listings}
\usepackage{gensymb}
\usepackage{array}
\usepackage{times}
\usepackage{xcolor}
\usepackage{default}
\usepackage{ulem}
\usepackage{minted}
\usepackage{hyperref}
\usepackage{booktabs}

% Great Commands
\newcommand{\ig}[2]{\includegraphics[width=#1\linewidth]{#2}}
\newcommand{\mybutton}[2]{\hyperlink{#1}{\beamerbutton{{#2}}}}
\newcommand{\myvbutton}[2]{\vfill\hyperlink{#1}{\beamerbutton{{#2}}}}
\newcommand{\code}[2]{\mintinline{#1}{#2}}
\newcommand{\python}[1]{\code{python}{#1}}
\newcommand{\bash}[1]{\code{bash}{#1}}
\newcommand{\unnamedUrl}[1]{\href{#1}{\color{blue}{#1}}}
\newcommand{\namedUrl}[2]{\href{#1}{\color{blue}{#2}}}
\newcommand{\pygment}[3]{\inputminted[bgcolor=lightgray,linenos,fontsize=#1]{#2}{#3}}
\newcommand{\pygmentLines}[5]{\inputminted[bgcolor=lightgray,linenos,fontsize=#1,firstline=#2,lastline=#3,autogobble]{#4}{#5}}

% Color Scheme
\definecolor{pittblue}{RGB}{28,41,87}
\definecolor{pittgold}{RGB}{205,184,125}
\setbeamercolor{structure}{fg=pittgold}
\setbeamercolor{button}{bg=pittblue}

\title[PyTorch]{{Introduction to PyTorch}}
\author[PyTorch]{{Barry Moore II}}
\institute[CRC]{Center for Research Computing \\ University of Pittsburgh}
\date{}

\beamertemplatenavigationsymbolsempty

\begin{document}

\begin{frame}[plain]
\titlepage
\end{frame}

\begin{frame}{Outline}
  \begin{itemize}
    \item Neurons and Automatic Differentiation
    \item A basic model from scratch
    \item PyTorch Dataset and DataLoader
    \item A basic dense neural net
    \item Convolutional neural nets and transfer learning
  \end{itemize}
\end{frame}

\begin{frame}{Neuron}
  \centering
  \ig{0.75}{figures/neuron.png}
  \begin{itemize}
      \item $f$ is a function which takes $x$ and $y$ as input and produces $z$
      \item The $...$ represents other neurons
  \end{itemize}
\end{frame}

\begin{frame}{Automatic Differentiation, How?}
  \centering
  \ig{0.5}{figures/autograd.png}
  \begin{itemize}
      \item Top. Forward pass
      \item Bottom. Backward pass
      \item Using chain rule,
        \begin{align*}
          \frac{\partial L}{\partial x} &= \frac{\partial L}{\partial z} \cdot
                    \frac{\partial z}{\partial x}
        \end{align*}
  \end{itemize}
\end{frame}

\begin{frame}{Automatic Differentiation, forward pass}
  \begin{itemize}
    \item Choose inputs and evaluate outputs
      \begin{align*}
        x &= 3 \\
        y &= 2 \\
        f(x, y) &= x + y \\
        z &= x + y = 5
      \end{align*}
  \end{itemize}
\end{frame}

\begin{frame}{Automatic Differentiation, backward pass}
  \begin{itemize}
    \item Remember inputs and complete local partial derivatives
      \begin{align*}
        \frac{\partial L}{\partial z} &= 10 \\
        \frac{\partial z}{\partial x} &= \frac{\partial x + y}{\partial x} = y \\
        \frac{\partial z}{\partial y} &= x \\
        \frac{\partial L}{\partial x} &= 10 * 2 = 20 \\
        \frac{\partial L}{\partial y} &= 10 * 3 = 30 \\
      \end{align*}
  \end{itemize}
\end{frame}

\begin{frame}{Automatic Differentiation, Exercise}
  \begin{itemize}
      \item Given:
        \begin{align*}
          f(x, w) &= x * w = \hat{y} \\
          L(\hat{y}, y) &= (\hat{y} - y)^2 = o
        \end{align*}
      \item $x$ is some input
      \item $w$ is a linear weight
      \item $\hat{y}$ is a prediction
      \item $L$ is a squared difference loss function
      \item $o$ is the output of the loss function
      \item Try writing out the directed acyclic graph
      \item Try computing $\frac{\partial L}{\partial w}$ when $x=2, w=1, y=4$
  \end{itemize}
\end{frame}

\begin{frame}{PyTorch by Example, Inputs}
  \centering
  \pygment{\scriptsize}{python}{code/basic-training.0.py}
  \ig{0.5}{figures/2x.png}
\end{frame}

\begin{frame}{PyTorch by Example, Forward and Loss}
  \pygment{\scriptsize}{python}{code/basic-training.1.py}
\end{frame}

\begin{frame}{PyTorch by Example, Training}
  \pygment{\scriptsize}{python}{code/basic-training.2.py}
  \vspace{-0.5cm}
  \begin{itemize}
      \item An epoch refers to one pass of the training data
      \item \code{python}{output.backward()} computes $\frac{\partial L}{\partial w}$
      \item Line 8. Adjust $w$ in the opposite direction of the gradient
      \item Line 9. Reset the gradients to zero for next epoch
      \item $0.001$ is the learning rate, choosing a large learning will cause
        the gradients to explode
  \end{itemize}
\end{frame}

\begin{frame}{Basic Steps}
  \begin{enumerate}
      \item Generate input data
      \item Generate your forward network (model)
      \item Train via back propagation
      \item Evaluate
  \end{enumerate}
\end{frame}

\begin{frame}{Step 1. Generate your input data}
  \begin{itemize}
    \item Start with a CSV file to describe input data
  \end{itemize}
  \pygmentLines{\scriptsize}{1}{5}{text}{data/mnist.csv}
  \begin{itemize}
    \item Handwritten image data set, e.g. \texttt{0.png}
  \end{itemize}
  \begin{center}
    \ig{0.25}{figures/0.png}
  \end{center}
\end{frame}

\begin{frame}{Building an example \python{Dataset}}
  \pygment{\scriptsize}{python}{code/basic-dataset.py}
\end{frame}

\begin{frame}{Building and example \python{DataLoader}}
  \pygment{\scriptsize}{python}{code/basic-dataloader.py}
  \vspace{-0.5cm}
  \begin{itemize}
    \item What is the type of $X$ and $y$?
    \item What is the length of $X$ and $y$?
  \end{itemize}
\end{frame}

\begin{frame}{Let's implement a custom dataloader}
  \begin{itemize}
    \item Try to implement a \python{Dataset} and \python{DataLoader} which
      compute the mean of the pixel values
    \item Result should be $33.79$
    \item Loading an image with Pillow:
  \end{itemize}
  \pygment{\scriptsize}{python}{code/pil-open.py}
\end{frame}

\begin{frame}{Mean and standard deviation implementation}
  \begin{itemize}
    \item Let's look at \texttt{notebooks/MeanAndStandardDeviation.ipynb} together
  \end{itemize}
\end{frame}

\begin{frame}{Building the MNIST inputs}
  \begin{itemize}
    \item mean: 33.79; standard deviation: 79.17
    \item Apply z-score to smooth out gradients
    \item \texttt{torchvision} has a module called \texttt{transforms}
    \begin{itemize}
      \item \python{transforms.Compose} concatenates transforms together
      \item \python{transforms.ToTensor} takes a PIL Image $(H, W, C)$
        in $[0, 255]$ and converts to Tensor $(C, H, W)$
      \item \python{transforms.Normalize} takes mean and standard deviation
        $(C)$ and applies z-score to all pixels
    \end{itemize}
  \end{itemize}
\end{frame}

\begin{frame}{Building the MNIST inputs}
  \begin{itemize}
    \item Let's look at \texttt{notebooks/MNISTVanilla.ipynb} together
  \end{itemize}
\end{frame}

\begin{frame}{Building a basic dense network}
  \begin{itemize}
    \item Let's build a basic neural net using \python{torch.nn.Sequential}
  \end{itemize}
  \pygment{\scriptsize}{python}{code/basic-dense-model.py}
\end{frame}

\begin{frame}{Optimizer and Loss Function}
  \begin{itemize}
    \item \namedUrl{https://arxiv.org/abs/1412.6980}{Adam Optimizer}
    \item \namedUrl{https://pytorch.org/docs/stable/nn.html\#crossentropyloss}{Cross Entropy Loss}
  \end{itemize}
  \pygment{\scriptsize}{python}{code/optimizer-loss-fn.py}
\end{frame}

\begin{frame}{Training the basic dense forward network}
  \begin{itemize}
    \item The number of ``epochs'' describes how many times the model sees the dataset
  \end{itemize}
  \pygment{\scriptsize}{python}{code/training-dense-model.py}
\end{frame}

\begin{frame}{Building a basic dense network: notebook}
  \begin{itemize}
    \item Let's look at \texttt{notebooks/BasicDense.ipynb} together
  \end{itemize}
\end{frame}

\begin{frame}{Convolutional neural net: a 2D convolution}
  \centering
  \ig{0.75}{figures/convolution.png}
  \begin{align*}
    \mathrm{kernel} &= (3, 3) \\
    \mathrm{stride} &= (1, 1) \\
    y_0 &= x_0 \times w_0 + w_1 \times x_0 + \dots \\
    y_1 &= x_1 \times w_0 + w_2 \times x_0 + \dots
  \end{align*}
\end{frame}

\begin{frame}{Convolutional neural net: transfer learning}
  \begin{itemize}
    \item ...
  \end{itemize}
\end{frame}

\begin{frame}{CNN via transfer learning}
  \begin{itemize}
    \item Most of the pretrained models require a 224x224 image
    \item \python{pretrained=False} will yield randomly initialized weights
  \end{itemize}
  \pygment{\scriptsize}{python}{code/transfer-model.py}
\end{frame}

\begin{frame}{CNN via transfer learning: notebook}
  \begin{itemize}
    \item Let's look at \texttt{notebooks/BasicTransfer.ipynb} together
  \end{itemize}
\end{frame}

\begin{frame}{Practical Considerations}
  \begin{itemize}
    \item Metrics should be on data held out from the training data
    \item Imbalanced datasets are extremely common, trivial upsampling:
    \begin{itemize}
      \item Generate max count $mc$, consider number of rows $nr$ for a label
      \item Duplicate rows up to $mc\ //\ r$
      \item Randomly sample rows up to $mc\ \%\ r$
      \item Combine duplicates and random samples into new dataset
    \end{itemize}
    \item Adding noise to images can help generalize models
      \namedUrl{https://pytorch.org/docs/stable/torchvision/transforms.html}{Torchvision
      transforms} has PIL \python{Image} transformations
  \end{itemize}
\end{frame}

\end{document}
